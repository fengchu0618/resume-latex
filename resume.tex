% !TEX TS-program = xelatex
% !TEX encoding = UTF-8 Unicode
% !Mode:: "TeX:UTF-8"
\documentclass{resume}
\usepackage{linespacing_fix} % disable extra space before next section
\usepackage{cite}


\begin{document}
\pagenumbering{gobble} % suppress displaying page number

{\heiti \name{ 胡 \hspace{0.1cm}正 \hspace{0.1cm}经 }}

% {E-mail}{mobilephone}{homepage}
% keep the last empty braces!
\contactInfo{hzj@example.com}{(+86) 188-888-8888}{}
% {E-mail}{mobilephone}
% keep the last empty braces!
%\contactInfo{xxx@yuanbin.me}{(+86) 131-221-87xxx}{}
 
\section{\faGraduationCap 教育背景}

\datedsubsection{\textbf{京师大学堂}, 北京,中国}{2016年9月 -- 2021年(预计)}
\textit{博士研究生}\ \ 计算机科学与技术

\datedsubsection{\textbf{清华学堂}, 北京, 中国}{2012年9月 -- 2016年7月}
\textit{本科}\ \ 软件工程

排名: 4 / 134 \ 成绩: 99 / 100 (主科目:99 / 100 )


\section{\faCogs\ IT 技能}
% increase linespacing [parsep=0.5ex]
\begin{itemize}[parsep=0.5ex]
  \item 研究兴趣:机器学习,概率图模型,数据挖掘,社交网络,多模态建模
  \item 编程语言:C / C++, Java, C\#, Python, PHP, JavaScript
  \item 编程框架:Node.js, Django, Android, WPF
  \item 科学软件:Matlab, Mathematics, Weka, openSMILE, LightSide
\end{itemize}


\section{\faUsers\ 科研 / 项目经历}
\datedsubsection{\textbf{X \& B实验室}, 北京大学, 中国}{2016年4月 -- 2016年8月}
\role{研究实习生}{导师: \ Prof.B}
\textbf{科研主题}: 这是第一个项目demo

\datedsubsection{\textbf{某教育部重点实验室}, 清华学堂, 中国}{2014年10月 -- 2016年4月}
\role{研究助教}{导师: \ A教授}
\textbf{科研主题}:这是第二个项目demo
\begin{itemize}
\item 提出一种基于\textbf{A}的方法
\item 改善了B问题
\item 基于以上发现构建了一个基于C模型
\end{itemize}


\section{\faHeartO\ 获奖情况}
\begin{itemize}
\item \datedline{国家奖学金(前 \textbf{2\%})}{2013 年 9 月}
\item \datedline{国家励志奖学金(前 \textbf{5\%})}{2014 年 9 月}
\item \datedline{美国大学生数学建模竞赛(MCM)一等奖}{2015 年 4 月}
\end{itemize}

\section{\faPaperPlane\ 其他信息}
\begin{itemize}
\item \textbf{个人主页}: http://github.io
\item \textbf{Github主页}: https://github.com
\end{itemize}


\end{document}
